% Math-specific preamble

\usepackage{amsmath} % provides many mathematical environments & tools
\usepackage{amsthm} % for proofs
\usepackage{amssymb} % math symbols
\usepackage{centernot} % for negations
\usepackage{mathtools}
\usepackage{graphicx}
\usepackage{parskip}
\usepackage[makeroom]{cancel} % math cancels
\usepackage{adjustbox}
\usepackage{breqn} % math equation break

\graphicspath{{./1-images/}}
\setlength{\parindent}{0pt}

% Set symbols shortcuts
\newcommand{\C}{\mathbb{C}}
\newcommand{\R}{\mathbb{R}}
\newcommand{\Q}{\mathbb{Q}}
\newcommand{\Z}{\mathbb{Z}}
\newcommand{\N}{\mathbb{N}}
\newcommand{\F}{\mathbb{F}}
\newcommand{\I}{\mathbb{I}}
\newcommand{\st}{\text{ s.t.\;}}
\newcommand{\eps}{\boldsymbol{\varepsilon}} % Also make epsilon bolder
\newcommand{\intr}[1]{\text{int}(#1)} % Interior points of set
\newcommand{\bd}[1]{\text{bd}(#1)} % Boundary points of set
\newcommand{\cl}[1]{\text{cl}(#1)} % Closure of set

\DeclareMathOperator*{\argmin}{argmin}
\DeclareMathOperator*{\argmax}{argmax}

\renewcommand{\implies}{\Rightarrow} % shorter implies
\renewcommand{\iff}{\Leftrightarrow} % shorter iff
\renewcommand{\qedsymbol}{\boxtimes} % Box with cross QED

\renewcommand{\Re}{\operatorname{Re}}
\renewcommand{\Im}{\operatorname{Im}}

% To write two-part definitions easier
\newcommand{\twopartdef}[4]
{
	\left\{
		\begin{array}{ll}
			#1 & #2 \\
			#3 & #4
		\end{array}
	\right.
}

% And three-part definitions
\newcommand{\threepartdef}[6]
{
	\left\{
		\begin{array}{lll}
			#1 & #2 \\
			#3 & #4 \\
			#5 & #6
		\end{array}
	\right.
}

% And four-part definitions
\newcommand{\fourpartdef}[8]
{
	\left\{
		\begin{array}{lll}
			#1 & #2 \\
			#3 & #4 \\
			#5 & #6 \\
			#7 & #8
		\end{array}
	\right.
}

% Underline vectors and double underline matrices
\newcommand{\vx}[1]{\underline{#1}}
\newcommand{\mx}[1]{\underline{\underline{#1}}}

% Better T for transpose
\newcommand{\transpose}[1]{#1^\intercal}

% More convenient abs
\newcommand{\abs}[1]{\left\lvert#1\right\rvert}

% More convenient floor and ceil
\DeclarePairedDelimiter\ceil{\lceil}{\rceil}
\DeclarePairedDelimiter\floor{\lfloor}{\rfloor}

% Linear algebra shortcuts
\newcommand{\rref}{\operatorname{rref}}
\newcommand{\mref}{\operatorname{ref}}
\newcommand{\vspan}{\operatorname{span}}
\newcommand{\vrank}{\operatorname{rank}}
\newcommand{\vnullity}{\operatorname{nullity}}
\newcommand{\vrange}{\mathcal{R}}
\newcommand{\vnull}{\mathcal{N}}
\newcommand{\norm}[1]{\left\lVert#1\right\rVert}
\DeclarePairedDelimiterX{\inner}[2]{\langle}{\rangle}{#1,#2}
\newcommand{\conj}[1]{\overline{#1}}

% Conditional probability
\newcommand\given{\:\vert\:}

% Probability & statistics shortcuts
\newcommand{\E}{\operatorname{E}} % Expectation
\newcommand{\Var}{\operatorname{Var}} % Variance
\newcommand{\Cov}{\operatorname{Cov}} % Covariance
\newcommand{\Bin}{\operatorname{Bin}} % Binomial dist
\newcommand{\Be}{\operatorname{Be}} % Bernoulli dist
\newcommand{\Po}{\operatorname{Po}} % Poisson dist
\newcommand{\Pa}{\operatorname{Pa}} % Pascal dist
\newcommand{\Exp}{\operatorname{Exp}} % Exponential dist
\newcommand{\Erl}{\operatorname{Erlang}} % Exponential dist
\newcommand{\Geo}{\operatorname{Geo}} % Geometric dist
\newcommand{\U}{\mathcal{U}} % Uniform dist
\newcommand{\Norm}{\mathcal{N}} % Normal dist
\newcommand*\mean[1]{\overline{#1}} % Mean (bar)
\newcommand{\pval}{p\text{-value}} % p-value

% Equalities
\newcommand{\eqdef}{\stackrel{\mathclap{\normalfont\mbox{def}}}{=}} % By definition
\newcommand{\eqdist}{\stackrel{\mathclap{d}}{=}} % Equal in distribution
\newcommand{\approxdist}{\stackrel{\mathclap{d}}{\approx}} % Approx in distribution

